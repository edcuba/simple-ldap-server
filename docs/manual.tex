\documentclass[12pt]{report}
\usepackage[a4paper]{geometry}
\usepackage[myheadings]{fullpage}
\usepackage{fancyhdr}
\usepackage{lastpage}
\usepackage{wrapfig, setspace}
\usepackage[slovak]{babel}
\usepackage[utf8]{inputenc}
\usepackage{url}

\newcommand{\HRule}[1]{\rule{\linewidth}{#1}}
\onehalfspacing
\setcounter{tocdepth}{5}
\setcounter{secnumdepth}{5}

%-------------------------------------------------------------------------------
% HEADER & FOOTER
%-------------------------------------------------------------------------------
\pagestyle{fancy}
\fancyhf{}
\setlength\headheight{15pt}
\fancyhead[L]{login: xcubae00}
\fancyhead[R]{Vysoké učení technické v Brně}
\fancyfoot[R]{Strana \thepage\ z \pageref{LastPage}}
%-------------------------------------------------------------------------------
% TITLE PAGE
%-------------------------------------------------------------------------------

\begin{document}

\title{ \normalsize \large \textsc{Dokumentácia}
        \\ [1.0cm]
        \LARGE \textbf{\uppercase{LDAP server}} \\
        [0.5cm]
        \normalsize \today \vspace*{5\baselineskip}}

\date{}

\author{
        Eduard Čuba \\
        login: xcubae00 \\
        Vysoké učení technické v Brně}

\maketitle
\tableofcontents

\chapter{Úvod}

Lightweight Directory Access Protocol (LDAP) je protokol pre ukladanie a
prístup k dátam na adresárovom serveri.
V rámci projektu z predmetu \emph{Sieťové aplikácie a správa sietí} som sa
venoval štúdiu tohoto protokolu a implementácii jednoduchého konkurentného
servera s podporou tohoto protokolu pre prístup k dátam.

Server poskytne prístup k dátam, ktoré sú pre zjednodušenie uložené v \texttt{CSV} súbore a
pre každý dátový objekt definujú atribúty:
\begin{itemize}
    \item \textbf{cn} - meno a priezvisko (\textbf{c}ommon \textbf{n}ame)
    \item \textbf{uid} - univerzitný login (\textbf{u}ser \textbf{id}) v tvare \texttt{xlogin00}
    \item \textbf{mail} - univerzitný email v tvare \texttt{xlogin00@stud.fit.vutbr.cz}
\end{itemize}

Server poskytuje podporu pre operáciu vyhľadania záznamov (\texttt{SearchRequest}) s využitím operácii
\texttt{and}, \texttt{or}, \texttt{not}, \texttt{equalityMatch} a \texttt{subString}.
Zvolený implementačný jazyk je \texttt{C++} s ohľadom na princípy objektového návrhu.

\section{Reprezentácia dát a kódovanie}

Pred samotným popisom protokolu je vhodné popísať formát, v akom sú správy zasielané.
Základom správ sú dátové objekty zodpovedajúce štandardu \texttt{ASN.1}\cite{ASN} (\textbf{A}bstract
\textbf{S}yntax \textbf{N}otation One), kódované podľa \texttt{BER}\cite{BER} (\textbf{B}asic
\textbf{E}ncoding \textbf{R}ules).

Pre potreby implementácie projektu budeme potrebovať následujúce dátové objekty.

\subsection{INTEGER}

Reprezentuje celé číslo v rozsahu $-32768$ až $32767$.

Typ je reprezentovaný hodnotou \texttt{Ox2}, následuje (minimálny) počet oktetov potrebných na
reprezentáciu čísla (1 až 4) a samotné dátové oktety.

\subsubsection*{Príklad}

\begin{itemize}
    \item Hodnota $42$ sa zakóduje ako \texttt{0x2 Ox1 Ox2a}
    \item Hodnota $420$ sa zakóduje ako \texttt{0x2 Ox2 Ox1 0xa4}
\end{itemize}


\subsection{Dĺžka}

Typ dĺžky je používaný ako súčasť viacerých dátových objektov.
Hodnote nepredchádza špecifikácia typu. Samotná dĺžka $L$ sa kóduje následujúcim spôsobom.

\begin{itemize}
    \item Ak $L < 128$, tak sa dĺžka zapíše na jediný oktet
    \item Ak $L >= 128$, tak hodnota prvého oktetu $H - 128$ reprezentuje počet nasledujúcich dátových oktetov
\end{itemize}

\subsubsection*{Príklad}

\begin{itemize}
    \item Dĺžka $100$ sa zakóduje ako \texttt{0x64}
    \item Dĺžka $159$ sa zakóduje ako \texttt{0x81 Ox9f}
\end{itemize}

\subsection{STRING}

Predstavuje reťazec oktetov. Reprezentuje ho hodnota \texttt{0x4}. Za hodnotou typu následuje počet
dátových oktetov vo formáte dĺžky a samotné dátové oktety.

\subsubsection*{Príklad}

\begin{itemize}
    \item \texttt{"xcubae00"} sa kóduje ako \texttt{0x4 0x8 0x78 0x63 0x75 0x62 0x61 0x65 0x30 0x30}
    \item \texttt{0x4} je dátový typ
    \item \texttt{0x8} je počet oktetov
    \item \texttt{0x78 0x63 0x75 0x62 0x61 0x65 0x30 0x30} sú ASCII hodnoty znakov
\end{itemize}

\subsection{SEQUENCE}

Sekvencia dátových objektov reprezentovaná hodnotou \texttt{0x30}, za ktorou následuje
dĺžka sekvencie a samotné dátové objekty.

\subsection{SET}

Skupina dátových objektov reprezentovaná hodnotou \texttt{0x31}, za ktorou následuje
dĺžka skupiny a samotné dátové objekty.

\chapter{Protokol LDAP}

Komunikáciu inicializuje klient zaslaním správy \texttt{bindRequest}, ktorá indikuje serveru,
že s ním klient chce naviazať spojenie.
V tejto časti je možné riešiť autentifikáciu užívateľa, čo však nie je predmetom projektu.

Server na požiadavku reaguje zaslaním správy \texttt{bindResponse}, v ktorej uvádza, či bolo
požiadavke na nadviazanie spojenia vyhovené.
V našom prípade odpovedáme na všetky požiadavky, ktorým server rozumie kladne.

Po úspešnom nadviazaní spojenia môže klient zasielať požiadavky na dátové objekty.
Požiadavku zašle v správe \texttt{searchRequest},
kde špecifikuje aké kritéria musia dátové objekty spĺňať.

Server požiadavku spracuje a ak sa v jeho databáze vyskytujú objekty,
ktoré vyhovujú zadaným kritériám, tak pre každý dátový objekt generuje
správu \texttt{searchResEntry}, ktorá obsahuje informácie o konkrétnom objekte.

Táto kapitola sa venuje popisu jednotlivých správ protokolu podporovaných serverom. Správy sú
bližšie popísané v dokumente \texttt{RFC 2251}\cite{RFC2251}.

\section{LDAP message}

Pre účely komunikácie sú objekty správy zaobalené v sekvencii \texttt{LDAPMessage}. Jej štruktúra
vyzerá v prípade nášho serveru následovne.

\begin{verbatim}
LDAPMessage ::= SEQUENCE {
    messageID       MessageID,
    protocolOp      CHOICE {
            bindRequest     BindRequest,
            bindResponse    BindResponse,
            searchRequest   SearchRequest,
            searchResEntry  SearchResultEntry,
            searchResDone   SearchResultDone,
            unbindRequest   UnbindRequest
    }
}
\end{verbatim}

\begin{itemize}
    \item messageID je identifikátor správy v rozsahu $0$ až $2^{31} -1$
    \item protocolOp je jedným z podporovaných typov správ.
\end{itemize}

\subsection{bindRequest}

Správa zaslaná klientom na zahájenie komunikácie.
Server nepodporuje autentifikáciu užívateľa pomocou \texttt{sasl},
podporovaná je výhradne metóda \texttt{simple}.

\subsection{bindResponse}

Správa odoslaná serverom na potvrdenie zahájenia spojenia vo formáte. Vzhľdom na to, že server
nepodporuje autentifikáciu by mal byť výsledok vždy \texttt{success (0)}.

\subsection{searchRequest}

Žiadosť klienta o informácie o dátových objektoch, ktoré spĺňaju kritéria špecifikované filtrami.

Medzi filtre podporované serverom patrí:
\begin{itemize}
    \item \texttt{and} - konjunkcia medzi viacerými filtrami
    \item \texttt{or} - disjunkcia medzi viacerými filtrami
    \item \texttt{not} - negácia jedného filtra
    \item \texttt{equalityMatch} - porovnanie reťazcov (bez ohľadu na veľké a malé písmená)
    \item \texttt{substrings} - zoznam podreťazcov
    \item \texttt{present} - všetky dostupné záznamy
\end{itemize}

\subsection{searchResEntry}

Správa obsahujúca jeden dátový objekt, ktorý vyhovuje špecifikovaným kritériám.
V prípade viacerých výsledkov sa pre každý dátový objekt generuje samostatná správa.

\subsection{searchResDone}

Správa oznamujúca, že boli zaslané všetky dátové objekty.

\subsection{unBind}

Žiadosť o ukončenie spojenia zasielaná klientom.

\chapter{Implementácia}

Server je implementovaný v jazyku \texttt{C++} s vo verzii \texttt{C++14}. Kód je formátovaný pomocou
utility \texttt{clang-format} a budovaný pomocou systému \texttt{make}.

\section{Štruktúra projektu}

Kód je rozdelený do 11 zdrojových a 11 hlavičkových súborov.
Rozdelenie funkcie zdrojových súborov je následovné:

\begin{itemize}
    \item \texttt{ber.cc} - kódovanie a dekódovanie dátových objektov podľa \texttt{BER}
    \item \texttt{cli.cc} - načítanie konfigurácie z príkazového riadku
    \item \texttt{csv.cc} - načítanie údajov z \texttt{CSV} súboru
    \item \texttt{dataset.cc} - operácie nad dátami v databáze (aplikácia filtrov)
    \item \texttt{ldap.cc} - stavový automat na príjem správy
    \item \texttt{filter.cc} - stavový automat na príjem filtrov, inicializácia objektu \texttt{ldapFilter}
    \item \texttt{message.cc} - objekt zapúzdrujúci LDAP správu
    \item \texttt{myldap.cc} - vstupný bod programu
    \item \texttt{response.cc} - objekt zapúzdrujúci odpoveď na dotaz (\texttt{ldapSearchResEntry})
    \item \texttt{result.cc} - objekt zapúzdrujúci výsledok (\texttt{ldapResult})
    \item \texttt{server.cc} - implementácia konkurentného servera a čítanie zo socketu
\end{itemize}

\section{Načítanie vstupných argumentov}

Načítanie argumentov z príkazového riadku je riešené manuálne jednoduchou smyčkou.

\section{Načítanie databázy}

Načítanie databázového súboru je riešené manuálne.
Vstupný súbor sa prechádza riadok po riadku s využitím \texttt{C++} triedy \texttt{std::ifstream}
a metódy \texttt{getline}. Riadok je rozdelený podľa znaku '\texttt{;}' na tri časti, ktoré sú
reprezentované ako meno, login a email.

Každý záznam je uložený v neusporiadanej mape (\texttt{std::unordered\_map}),
podľa kľúčov \texttt{cn}, \texttt{uid/userid} (podporované sú oba tvary) a \texttt{mail}.
Jednotlivé záznamy (neusporiadené mapy) sú uložené do poľa (\texttt{std::vector}).

\section{Konkurentný server}

Server vytvorí TCP socket na požadovanom porte (predvolený je 389) pomocou systémových volaní
\texttt{socket}, \texttt{bind} a \texttt{listen}.
Následne čaká na prichádzajúce spojenie volaním \texttt{accept}.
Po prijatí požiadavku vytvorí nové obslužné vlákno (s využitím \texttt{std::thread}) a znova sa
vracia do stavu čakania na prichádzajúce spojenie.

\section{Spracovanie správy}

Obslužné vlákno spustí spracovanie správy pomocou stavového automatu.
Správa sa číta postupne po jednom bajte.
Najprv sa vytvorí objekt \texttt{ldapContext}, ktorý drží informácie o správe.
Automat začína spracovaním dĺžky a identifikátora správy.

Následuje voľba protokolu, kde sa stavy vetvia podľa druhu požiadavku.
\begin{itemize}
    \item \texttt{bindRequest/searchRequest} presunie automat do stavu príjmu správy
    \item \texttt{unBind} ukončí spracovanie, uzavrie socket a ukončí činnosť vlákna
\end{itemize}

\section{Generovanie odpovede}

Po prijatí celej správy sa spustí proces generovanie odpovede.
\begin{itemize}
    \item \texttt{bindRequest} - generuje správu \texttt{bindResponse}
    \item \texttt{searchRequest} - aplikuje filtre na databázu, pre každý výsledok
        generuje správu typu \texttt{searchResEntry} a následne odošle správu \texttt{searchResDone}.
\end{itemize}

\section{Filtre}

Filtre sú reprezentované ako rekurzívna dátová štruktúra.
\begin{verbatim}
ldapFilter {
    filterType          type
    string              attributeDesc
    string              assertionValue
    vector<ldapFilter>  subFilters
    vector<subString>   subStrings
}
\end{verbatim}
Vyhodnotenie začína aplikáciou koreňového filtra na mäkkú kópiu databázy (ďalej len \texttt{dataset}).
Spôsob vyhodnotenia filtra závisí od jeho typu.
\begin{itemize}
    \item \texttt{equalityMatch/substrings} - v datasete ponechá len záznamy ktoré vyhovujú podmienke filtra
    \item \texttt{not} - zavolá negovaný filter s kópiou vstupného datasetu a vráti doplnok výsledku vzhľadom k vstupnému datasetu
    \item \texttt{and} - aplikuje na vstupný dataset všetky podfiltre
    \item \texttt{or} - aplikuje podfiltre na kópie vstupného datasetu a vráti ich zjednotenie
\end{itemize}

\section{Chybové stavy}

V prípade chybového stavu pri spracovaní správy sa spracovanie preruší, vypíše sa chyba na
\texttt{stderr}, socket sa uzavrie a vlákno sa ukončí.

\section{Rozšírenia}

\begin{itemize}
    \item Podpora filtra \texttt{present} - pre jednoduchšie testovanie a vypísanie všetkých záznamov
    \item Podpora národných znakov v kódovaní \texttt{UTF-8}
\end{itemize}

\chapter{Testovanie}

Server bol testovaný manuálne pomocou utility \texttt{ldapsearch} z balíku \texttt{OpenLDAP}
s dátovým súborom v kódovaní \texttt{UTF-8} so SR/ČR národnými znakmi.

Ako testovacie prostredie bol použitý operačný systém \texttt{Fedora 26, Fedora 27} a
\texttt{CentOS 7.4.1708 (merlin)} s kompilátorom \texttt{GCC 7.2.1} a \texttt{GCC 6.4 (merlin)}.

\begin{thebibliography}{9}
    \addcontentsline{toc}{chapter}{Literatúra}

    \bibitem{RFC2251}
        Wahl, M., Howes, T., and S. Kille,
        \textit{Lightweight Directory Access Protocol (v3)},
        RFC 2251,
        December 1997.

    \bibitem{ASN}
        Emmanuel Lécharny,
        \textit{Ldap ASN.1 Codec},
        Október 2006,
        \\\url{https://cwiki.apache.org/confluence/display/DIRxSRVx10/Ldap+ASN.1+Codec}
        \\Navštívené 12. Novembra 2017.

    \bibitem{BER}
        Burton S. Kaliski Jr., Ph.D.,
        RSA Data Security, Inc. Public-Key Cryptography Standards (PKCS),
        An RSA Laboratories Technical Note,
        \textit{A Layman's Guide to a Subset of ASN.1, BER, and DER},
        November 1, 1993,
        \\\url{http://luca.ntop.org/Teaching/Appunti/asn1.html}
        \\Navštívené 12. Novembra 2017.

\end{thebibliography}

\end{document}
